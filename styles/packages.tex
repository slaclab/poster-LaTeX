% This works both with inline lists and with macros containing lists
\newcommand*{\GetListMember}[2]{%
    \edef\dotheloop{%
    \noexpand\foreach \noexpand\a [count=\noexpand\i] in {#1} {%
        \noexpand\IfEq{\noexpand\i}{#2}{\noexpand\a\noexpand\breakforeach}{}%
    }}%
    \dotheloop
    \par%
}%

% https://tex.stackexchange.com/questions/184005/adding-arrows-to-a-gantt-chart
% GanttHeader setups some parameters for the rest of the diagram
% #1 Width of the diagram
% #2 Width of the space reserved for task numbers
% #3 Width of the space reserved for task names
% #4 Number of months in the diagram
% #5 Initial month number
% In addition to these parameters, the layout of the diagram is influenced
% by keys defined below, such as y, which changes the vertical scale
\def\GanttHeaderIntraYear#1#2#3#4#5{%
 \pgfmathparse{(#1-#2-#3)/#4}
 \tikzset{y=7mm, task number/.style={left, font=\bfseries},
     task description/.style={text width=#3,  right, draw=none,
           % font=\sffamily, xshift=#2,
           xshift=#2,
           minimum height=2em},
     gantt bar/.style={draw=black, fill=blue!30}, % default color
     help lines/.style={draw=black!50, dashed},
     x=\pgfmathresult pt
     }
  \def\totalmonths{#4}
  \def\monthsname{Jan,Feb,Mar,Apr,May,Jun,Jul,Aug,Sep,Oct,Nov,Dec}
  % \node (Header) [task description] at (0,0) {\textbf{\large Tasks}};
  \node (Header) [task description] at (0,0) {};
  \begin{scope}[shift=($(Header.south east)$)]
	  	\node[above, anchor=north] at (0,0.75) {\GetListMember{\monthsname}{1}};
	  \foreach[count = \z from #5+1, count = \x from #5] \y in {#5,...,#4}
	  	\node[above, anchor=north] at (\x,0.75) {\GetListMember{\monthsname}{\z}};
 \end{scope}
}


\def\GanttHeader#1#2#3#4#5{%
 \pgfmathparse{(#1-#2-#3)/#4}
 \tikzset{y=7mm, task number/.style={left, font=\bfseries},
     task description/.style={text width=#3,  right, draw=none,
           % font=\sffamily, xshift=#2,
           xshift=#2,
           minimum height=2em},
     gantt bar/.style={draw=black, fill=blue!30},
     help lines/.style={draw=black!50, dashed},
     x=\pgfmathresult pt
     }
  \def\totalmonths{#4}
  \def\monthsname{Jan,Feb,Mar,Apr,May,Jun,Jul,Aug,Sep,Oct,Nov,Dec}
  % \node (Header) [task description] at (0,0) {\textbf{\large Tasks}};
  \node (Header) [task description] at (0,0) {};
  \begin{scope}[shift=($(Header.south east)$)]
	  \foreach[count = \x from 0] \y in {#5,...,12}
	  	\node[above, anchor=north] at (\x,0.75) {\GetListMember{\monthsname}{\y}};
	  % \node[above] at (\x,1) {\Large{\y}};
    \foreach \x in {1,...,#4}
	  \node[above, anchor=north] at (\x,0.75) {\GetListMember{\monthsname}{\x}};
 \end{scope}
}


\newcommand\Task[5]{
	\Taskk{#1}{#2}{#3}{#4}{blue}{#5}
}


\def\Taskk#1#2#3#4#5#6{%
\node[task number] at ($(Header.west) + (0, -#1)$) {\normalfont#6};
\node[task description] at (0,-#1) {#2};
\begin{scope}[shift=($(Header.south east)$)]
  \draw (0,-#1) rectangle +(\totalmonths, 1);
  \foreach \x in {1,...,\totalmonths}
    \draw[help lines] (\x,-#1) -- +(0,1);
  \filldraw[gantt bar, fill=#5!30] ($(#3, -#1+0.2)$) rectangle +(#4,0.6);
\end{scope}
}

\def\Range#1#2#3#4#5#6{%
\node[task number] at ($(Header.west) + (0, -#1)$) {#6};
\node[task description] at (0,-#1) {\bfseries#2};
\begin{scope}[shift=($(Header.south east)$)]
  \draw (0,-#1) rectangle +(\totalmonths, 1);
  \foreach \x in {1,...,\totalmonths}
    \draw[help lines] (\x,-#1) -- +(0,1);
  \filldraw[gantt bar, fill=#5!30] ($(#3, -#1+0.4)$) rectangle +(#4,0.2);
\end{scope}
}





%%%%%%%%%%%%%%%%%%%%%%%
% Josephson junction

\makeatletter
\pgfcircdeclarebipolescaled{instruments}
{
    % put the node text above and centered
    \anchor{text}{\pgfextracty{\pgf@circ@res@up}{\northeast}
        \pgfpoint{-.5\wd\pgfnodeparttextbox}{
            \dimexpr.5\dp\pgfnodeparttextbox+.5\ht\pgfnodeparttextbox+\pgf@circ@res@up\relax
        }
    }
}
{\ctikzvalof{bipoles/oscope/height}}
{josephson}
{\ctikzvalof{bipoles/oscope/height}}
{\ctikzvalof{bipoles/oscope/width}}
{
    \pgf@circ@setlinewidth{bipoles}{\pgfstartlinewidth}
    \pgfextracty{\pgf@circ@res@up}{\northeast}
    \pgfextractx{\pgf@circ@res@right}{\northeast}
    \pgfextractx{\pgf@circ@res@left}{\southwest}
    \pgfextracty{\pgf@circ@res@down}{\southwest}
    \pgfmathsetlength{\pgf@circ@res@step}{0.25*\pgf@circ@res@up}
    \pgfscope
        \pgfpathrectanglecorners{\pgfpoint{\pgf@circ@res@left}{\pgf@circ@res@down}}{\pgfpoint{\pgf@circ@res@right}{\pgf@circ@res@up}}
        \pgf@circ@draworfill
    \endpgfscope
    \pgfscope
      \pgfpathmoveto{\pgfpoint{\pgf@circ@res@left}{\pgf@circ@res@up}}%
      \pgfpathlineto{\pgfpoint{\pgf@circ@res@right}{\pgf@circ@res@down}}%
      \pgfpathmoveto{\pgfpoint{\pgf@circ@res@right}{\pgf@circ@res@up}}%
      \pgfpathlineto{\pgfpoint{\pgf@circ@res@left}{\pgf@circ@res@down}}%
      \pgfusepath{draw}
    \endpgfscope
}
\def\pgf@circ@josephson@path#1{\pgf@circ@bipole@path{josephson}{#1}}
\tikzset{josephson/.style = {\circuitikzbasekey, /tikz/to path=\pgf@circ@josephson@path, l=#1}}



% Saturator block
\tikzset{%
  saturation block/.style={%
    draw,
    path picture={
      % Get the width and height of the path picture node
      \pgfpointdiff{\pgfpointanchor{path picture bounding box}{north east}}%
        {\pgfpointanchor{path picture bounding box}{south west}}
      \pgfgetlastxy\x\y
      % Scale the x and y vectors so that the range
      % -1 to 1 is slightly shorter than the size of the node
      \tikzset{x=\x*.4, y=\y*.4}
      %
      % Draw annotation
      \draw (-1,0) -- (1,0) (0,-1) -- (0,1);
      \draw (-1,-.7) -- (-.7,-.7) -- (.7,.7) -- (1,.7);
    }
  }
}



